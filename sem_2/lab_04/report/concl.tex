\chapter{Обоснование использования специальных функции}

В Linux каждый процесс имеет собственное изолированное адресное пространство, то есть указатель ссылается не на уникальную позицию в физической памяти, а на позицию в адресном пространстве процесса.

Во время выполнения обычной программы адресация происходит автоматически, если же выполняется код ядра и необходимо получить доступ к странице кода ядра, то необходим буфер.

Когда мы хотим передавать информацию между процессом и кодом ядра, то соответсвующая специальная функция (copy\_from\_user, copy\_to\_user) получает указатель на буфер процесса.
