\Introduction

\textbf{Цель:} углубленное изучение системных вызовов на примере системного вызова open().

\textbf{Задание:}
Построить схему выполнения системного вызова open() в зависимости от значения основных флагов определяющих открытие файла на чтение, на запись, на выполнение и на создание нового файла. В схеме должны быть названия функций и кратко указаны выполняемые ими действия. По ГОСТу это делается с помощью выносных линий в фигурных скобках.

В схему нужно обязательно включить следующие действия, выполняемые соответствующими функциями ядра:
\begin{itemize}
    \item копирование названия файла из пространства пользователя в пространство ядра;

    \item блокировка/разблокировка (spinlock) структуры files\_struct и других действий в разных функциях;

    \item алгоритм поиска свободного дескриптора открытого файла;

    \item работу со структурой nameidata – инициализация ее полей;

    \item алгоритм разбора пути (кратко);

    \item инициализацию полей struct file;

    \item «открытие» файла для чтения, записи или выполнения;

    \item создание inode в случае отсутствия открываемого файла.
\end{itemize}
